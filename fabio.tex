\documentclass{iucr}

\begin{document}

\title{FabIO: easy access to area X-ray detector images in Python}
    \author[a]{Henning O.}{S\orensen}
    \author[a]{Erik B.}{Knudsen}
    \author[b]{Jonhatan P.}{Wright}
	\cauthor[b]{J\'er\^ome}{Kieffer}{jerome.kieffer@esrf,fr}{}
	\author[b]{Ga\"el}{Goret}
	\author[b]{V. Armando}{Sol\'e}
	
    \aff[a]{Institut of chemistry, \city{Copenhagen}, \country{Denmark}}
    \aff[b]{European Synchrotron Radiation Facility, \city{Grenoble}, \country{France}}

     % Use \shortauthor to indicate an abbreviated author list for use in
     % running heads (you will need to uncomment it).
\maketitle

\section{Introduction}

One obstacle when writing programs to analyse X-ray diffraction data or other
data collected from a two-dimensional detector is to read the raw data into the
program, not least because the data can be stored in many different formats
depending on the instrument used. 
We have had a similar problems and therefore decided to develop a general module, 
FabIO (FABle I/O), for easy reading of two-dimensional data. 
The module has been developed as part of the FABLE (Fully Automated BeamLine 
XXX) program suite for analysis of 3DXRD microscopy
data\cite{3dxrd}. The module is written in object-oriented scripting language
Python\cite{python} . Table \ref{format} shows the fairly comprehensive list of
file formats that Fabio can currently (ver. 0.1.0) read. FitAllB\cite{fitallb}.


\section{FabIO python module} 

\subsection{Philosophy}

The basic idea was to make a python module, which would enable very straight
forward reading of 2D diffraction  data images from any detector format
without having to worry about the format.
Therefore Fabio just needs a file name to open a file and make out the format. 
The data should be stored in memory as a standard 2D numpy array\cite{numpy}.
If the file contains image header information which should also be read and 
stored as a standard python dictionary. 
Furthermore there should be functions for reading the previous or next image 
in a series of images as well as jumping to a specific file number. 
For the user auxiliary functions should be independent of the image format.
The software should freely available to everyone and it should be modular in 
order to ensure the possibility for including other data formats. 
Fabio is licensed as GNU General Public License version 3 (GPLv3) 
and the source code is freely available on-line\cite{fabio}.


\subsection{Implementation}

Generally Fabio have been code in Python\cite{python},
though for some image formats the python code is merely interfaced a reading 
layer programmed in standard C (sometimes using cython\cite{cython}.
Fabio makes use of the numerical python module numpy\cite{numpy} and there are
two optional python modules which can be used Lxml, for reading XML files
used in EDNA\cite{edna} and Python Image Library
\cite{pil}, for converting the image to be visualized in graphical user
interfaces based on the Python Tkinter library\cite{???} or simply with matplotlib\cite{matplotlib}. 
In the Python shell the FabIO module must be imported before reading an image of any of the supported formats, see Table \ref{format},  
this creates an instance of the python class {\em fabioimage} in this example called {\em img}.
This instance stores the image data, but also includes a number of methods.
im.data now has all the pixel values in the image stored as a numpy array. 
Often the image file contains furthermore information than just the intensities of the pixels, 
e.g. information about how the image is stored and the instrument parameters at the time of the image acquisition.
Header information, if available, is available in im.header as a Python dictionary.
 
\begin{table}[h]
\caption{\label{format}List of file formats that Fabio can read and write}
\vspace{1mm}
\begin{center}
\begin{tabular}{lccc}
Name 			& File extension & Multi-image	& Write\\% & Notes\\
\hline
ESRF data format	&	.edf	&	Yes	&	Yes		\\%Several bit formats\\
Nonius 	KappaCCD	&	.kccd	&	No 	&	No		\\%Multiples frames are merged are reading. now bruker\\
Bruker formats		&	.sfrm	&	?	&	?		\\%TODO Complete\\
Oxford Diffraction 	&	.oxd	&	No	&	Yes		\\%now Agilent\\
ADSC Quantum		&	.adsc	&	No?	&	?		\\%Check\\	
General Electric	&	-		&	Yes	&	No?		\\%massivly muflit-frame\\
Dectris Pilatus		&	.tif	&	No?	&	No?		\\%Tiff based format\\
MarCCD/Mar165		&	.mccd	&	No?	&	No?		\\%Tiff based format\\
Mar345				&	-		&	No	&	Yes		\\%Compressed format\\
Hamamatsu HiPiC		&	.tif	&	No?	&	No?	 	\\%Tiff based format\\
CIF Binary Files	&	.cbf	&	No?	& 	Yes		\\%Used also by Pilatus\\
DM3					&	.dm3	&	?	&	?		\\%?\\
PNM					&	.pnm	&	?	&	?		\\%?\\
TIFF				&	.tif	&	?	&	?		\\%can be compressed or not\\
EDNA\cite(edna}-XML	&	.xml	&	No	&	No	 \\%Internaly used in EDNA\cite{edna}\\
\end{tabular}
\end{center}

\end{table}

\begin{verbatim}
> import fabio 								# import the fabio module 
> img = fabio.open(‘exampleimage0001.edf’) 	# Create a fabioimage instance from the file
> img.header 								# prints all headers from the file as a python dict. 
{'ByteOrder': 'LowByteFirst',
 'DATE (scan begin)': 'Mon Jun 28 21:22:16 2010',
 'DataType': 'UnsignedShort',
 'Detector': 'frelon4m (sn=29)',
 'Dim_1': '2048',
 'Dim_2': '2048',
 'ESRFAutoTime': '62.2611',
 'ESRFBeamLine': 'id11',
 'ESRFCurrent': '200.099',
 'ESRFFillMode': '7/8 multibunch',
 'ESRFRefill': '38265',
 'Experiment': 'ma558',
 'Fz1': '20.465713',
 'Fz2': '20.467421',
……
 'zb': '22.300000',
 'zm': '1.239060'}
> Fz1 = img.header[‘Fz1’]					# retrieve an individual value
>
> import pylab								# import pylab from matplotlib  
> pylab.imshow(img.data)					# display as an image
> pylab.show()								# show window
\end{verbatim}

\subsection{FabIO methods}

One the strengths of the implementation in an object oriented language as Python
is the possibility to ship functions (or methods) togeather with data. So in
addition to the header information and the image data, every fabioimage instance
(returned by fabio.open) provides methods which provide information about the image – minimum, maximum, mean values, 
methods which gives the file number, name etc. One of the most important method
is to read data from another image into the instance, these functions are next, previous, and getframe.
   

\subsection{Examples}
\begin{verbatim}
> import fabio 							
> im = fabio.open(’Quartz_0100.tif’)	# Open image file
> im.data								# Print image data								  
> im = im.next()						# Open next image
> im = im.getframe(270)				# Jump to file number.
\end{verbatim} 
\subsection{acknowledgement}
We acknowledge Andy G\"otz, Kenneth Evans for including the Fabio reader in the
Fable ImageViewer.  
Carsten Gundlach for implementation of Fabio at the beamlines i711 and i811, MAX IV and providing bug reports.
We finaly acknowledge all people who reported bug on Fabio. 
Financial support was granted by the EU 6th Framework NEST/ADVENTURE project
TotalCryst.

\bibliographystyle{iucr}
\bibliography{biblio}






\end{document}
